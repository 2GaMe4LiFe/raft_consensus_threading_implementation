\chapter{Installation von Sobjectizer}

\vspace{10mm}

Sobjectizer ist ein Framework für C++ welches das Verwenden des Actor-Models vereinfachen soll.
Das einbinden von diesem Framework in mein Projekt geschieht über eine Shared-Library.
Um Sobjectizer jedoch verwenden zu können ist es erst einmmal notwendig die Libraries zu kompilieren.

\section{Kompilieren von Sobjectizer}
Für das kompilieren von Sobjectizer wird von den Herstellern Stiffstream CMake vorausgesetzt.
Das builden sieht unter Linux folgendermaßen aus:

\begin{minted}{bash}
git clone https://github.com/stiffstream/sobjectizer
cd sobjectizer
mkdir cmake_build
cmake -DCMAKE_INSTALL_PREFIX=target -DCMAKE_BUILD_TYPE=Release ../dev
cmake --build . --config Release
cmake --build . --config Release --target install
\end{minted}

Nun sollten sich in einem neu erstellten target-Folder zwei Ordner befinden. Einmal include und einmal
lib. In dem lib-Folder befinden sich die kompilierte Static- und Dynamic-Library. Im include-Folder können
alle notwendigen header-Files gefunden werden.

\section{Setup zum Verwenden des Projekts}
