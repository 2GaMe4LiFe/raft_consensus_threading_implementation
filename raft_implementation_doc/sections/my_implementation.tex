\chapter{Meine Raft-Implementation}

\vspace{10mm}

\section{Server}
Wie es in Raft üblich ist bestehen meine ''Server'' aus drei States: leader, follower und candidate.
zusätzlich gibt es einen vierten State, welcher den Ausfall eines Servers simulieren soll.

\begin{minted}{c++}
state_t server_state{this},
candidate{substate_of(server_state)},
follower{initial_substate_of(server_state)},
leader{substate_of(server_state)},
inactive{substate_of(server_state)};
\end{minted}

Für den eigentlichen Nachrichtenaustausch wurden verschiedene Klassen als Container definiert.

\subsection{ClientRequest}
\begin{minted}{c++}
    struct ClientRequest {
        so_5::mbox_t inbox;
        std::string cmd;
    };
\end{minted}
Der ClientRequest wird vom Client an einen Server geschickt. Dieser Request enthält die Information
wohin der Response geschickt werden muss: ''inbox''. Für die Serverseite nicht weniger wichtig ist
aber das Attribut ''cmd'', welches den Befehl enthält, welcher von diesem verarbeitet werden soll.

\subsection{ServerResponse}
\begin{minted}{c++}
    struct ServerResponse {
        int status; // 0 or 1 if 0 then request answered by follower
        std::string resp;
        std::string leader;
    };
\end{minted}
Der Serverresponse ist das Gegenstück zum ClientRequest. Der Server sendet diesen mit den  Informationen
wer den Response gesendet hat, was das Ergebnis der Verarbeitung ist und wer der derzeitige Leader des 
Clusters ist mit. Die Information über den Leader ist deswegen wichtig, da der Client beim ersten Request
eine Anfrage an einen zufälligen Server im Cluster sendet, welcher dann den Client über den derzeitigen
Leader informieren muss. Um zwischen dem Response eines Leaders und eines Followers zu unterscheiden gibt
es das Attribut ''status''.

\subsection{SetCluster}
\begin{minted}{c++}
    struct SetCluster {
        std::unordered_map<std::string, so_5::mbox_t> mboxes;
    };
\end{minted}
Diese Nachricht ist dafür zuständig die einzelnen Teilnehmer und den Client darüber zu informieren, welcher
Server wie zu erreichen ist und stellt somit den Beginn des Algorithmus dar.

\subsection{deactivate}
\begin{minted}{c++}
    struct deactivate : public so_5::signal_t{};
\end{minted}
''deactivate'' ist ein Signal, welches den Server einfach nur in einen extrigen State ''inactive'' versetzt oder herausholt,
je nach dem ob der server sich gerade in diesem befindet. Dieser State simuliert den Ausfall eines Servers.

\subsection{RequestVote}
\begin{minted}{c++}
    struct RequestVote {

        RequestVote(int term_, bool vote_granted_) : term{term_},
        vote_granted{vote_granted_} {}

        RequestVote(int term_, std::string candidate_id_, int lli_, int llt_)
        : term{term_}, candidate_id{candidate_id_}, last_log_index{lli_},
        last_log_term{llt_}, vote_granted{false} {}

        int term;
        std::string candidate_id;
        int last_log_index;
        int last_log_term;

        bool vote_granted;
    };
\end{minted}
Dieses struct ist so aufgebaut wie es im Raft-Paper spezifiziert ist. Diese Nachricht wird nur aus dem
candidate-State verschickt. Wenn mehr als die hälfte der im Cluster befindlichen Server mit einem positiven ''vote\_granted''
wird der Server vom candidate- in den leader-State versetzt.

\subsection{AppendEntry}
\begin{minted}{c++}
    struct AppendEntry {
        
        AppendEntry(int term_, bool success_) : term{term_},
        success{success_} {};

        AppendEntry(int term_, std::string leader_id_, int prev_log_index_,
        int prev_log_term_, std::vector<std::tuple<int,std::string>> entries_,
        int leader_commit_) : term{term_}, leader_id{leader_id_},
        prev_log_index{prev_log_index_}, prev_log_term{prev_log_term_},
        entries{entries_}, leader_commit{leader_commit_} {}
        
        int term;
        std::string leader_id;
        int prev_log_index;
        int prev_log_term;
        std::vector<std::tuple<int,std::string>> entries;
        int leader_commit;
        bool success;
    };
\end{minted}
AppendEntry ist ebenfalls so aufgebaut wie es im Raft-Paper spezifiziert wurde. Da es aber Probleme bei der
Umsetzung gab musste eine zusätzlicher Nachricht ''AppendEntryResult'' hinzugefügt werden. ''AppendEntry'' dient nicht
nur dazu das Log des leaders zu verteilen sondern auch dazu die follower davon abzuhalten ein Timeout zu erfahren.
Für diesen ''Heartbeat'' wird das Attribut ''entries'' leergelassen.

\subsection{AppendEntryResult}
\begin{minted}{c++}
    struct AppendEntryResult {
        std::string follower_name;
        int term;
        bool success;
    };
\end{minted}
Wie bereits oben erwähnt gab es Probleme bei der implementierung der Antwort auf AppendEntry-Nachrichten. Deshalb wurde ein
struct hinzugefügt, welches das Prozedere vereinfachen sollte.

\subsection{Heartbeats}
Heartbeats werden vom leader in einem Thread ausgeführt, welcher vom Hauptprogramm getrennt wird(detached). Mithilfe einer
Variable kann die darin enthaltene Loop angehalten werden. In der Loop selbst werden in einem bestimmten Zeitraum immer wieder
''AppendEntries'' als Heartbeat ausgesand, damit die follower kein Timeout erleben.
