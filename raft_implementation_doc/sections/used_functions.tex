\chapter{Benutzte Sobjektizer-Funktionen}

\vspace{10mm}

SObjectizer besitzt an sich eigentlich eine Dokumentation. Leider ist diese
nicht sonderlich detailreich und in manchen Fällen auch nicht aufschlussreich.
Der einzige Weg herauszufinden wie etwas funktioniert war es in einem Beispiel auszuprobieren.

\section{Funktionen}

\subsection{so\_5::launch(\textit{[](so\_5::environment\_t\&)\{\})}}
Diese Funktion ist essentiell für die Verwaltung von Agents. Prinzipiell wird nur ein Argument übernommen.
Dieses ist eine Funktion welche eine so\_5::environment\_t Instanz verwaltet. Auf Basis dieser Instanz werden
die Agents angelegt und verwaltet. Erst wenn diese Umgebung beendet wird wird die launch-Funktion beendet.

\subsection{introduce\_coop(\textit{[](so\_5::coop\_t\&)\{\})}}
introduce\_coop() ist eine Member-Function von so\_5::environment\_t. Diese Funktion wird in der Dokumentation von
SObjectizer nicht erklärt. Sie wird einfach als Code-Snippet als notwendiges Übel bereitgestellt. Der einzige
Parameter ist wieder wie in so\_5::launch() eine Funktion welche sich wiederum um das eigentliche Anlegen der
Agents kümmert. Der Parameter der übergebenen Funktion ist ein so\_5::coop\_t Objekt.

\subsection{make\_agent$<$\textit{Klasse}$>$(\textit{Paramliste des Konstruktors der Klasse})}
Die verwendete Klasse muss immer von so\_5::agent\_t abgeleitet sein um make\_agent verwenden zu können.
Es wird ein Zeiger auf die Klasse angelegt. Der einfachste Weg das Resultat zu speichern ist ''auto'' zu
verwenden.

\subsection{so\_direct\_mbox()}
Im Kontext zu meinem Programm ist dies die wichtigste Funktion. Mit ihr wird die Inbox des erstellten Agents
zurückgeliefert. Mithilfe dieser können wie bereits erwähnt die einzelnen Agents miteinander Kommunizieren.
Wenn der Agent mittels make\_agent aufgerufen wurde ist es notwendig darauf zu achten, dass das erstellte Objekt
ein Pointer ist.

\subsection{so\_5::send$<$\textit{Klasse/Struct}$>$(so\_5::mbox\_t, \textit{Paramliste des Konstruktors der Klasse})}
Der erste Parameter ist immer die Inbox des Agents den man ansprechen möchte. Alle weiteren Parameter werden
dazu genutzt die Klasse zu bilden. Die Funktion kann sowohl innerhalb von launch als auch in diversen Agents
verwendet werden. Mithilfe dieser Funktion wird die Kommunikation der Agents ermöglicht. Die Klasse wird als
''Brief'' verwendet.

\subsection{on\_enter(\textit{Funktion})}
Diese Funktion kann in Kombination von States verwendet werden. Wenn der State eines Agent zum spezifizierten State
geändert wird, wird sofort der Code aus der übergebenen Funktion ausgeführt. Das kann benutzt werden um Variablen zu
initialisieren oder um anderen Agents mitzuteilen, dass sich dieser jetzt in jenem State befindet.

\begin{minted}{c++}
class test final : public so_5::agent_t{
public:

  void so_define_agent() override {
  
    off.on_enter([this]{
      cout << "Dieser Agent ist gerade inaktiv geworden!!!" << endl;
    });
    
    on.on_enter([this]{
      cout << "Dieser Agent ist gerade aktiv geworden!!!" << endl;
    });
  
  }

private:
  state_t server_state,
  on{substate_of(server_state)},
  off{initial_substate_of(server_state)};
}
\end{minted}

Als Eklärung: ''so\_define\_agent()'' ist eine Funktion, mit welcher das Verhalten eines
Agents definiert wird. Diese Funktion ist von der Parent-Class ''agent\_t'' vererbt worden
und muss überschrieben werden.

\subsection{on\_exit(\textit{Funktion})}
Ist das Gegenstück zu ''on\_enter'' und wird aufgerufen wenn der State verlassen wird und kann somit als Destruktor
gesehen werden.

\begin{minted}{c++}
class test final : public so_5::agent_t{
public:

  void so_define_agent() override {
  
    off.on_exit([this]{
      cout << "Dieser Agent ist gerade dabei in den Dienst zu wechseln!!!" << endl;
    });
    
    on.on_exit([this]{
      cout << "Dieser Agent beendet gerade seinen Dienst!!!" << endl;
    });
  
  }

private:
  state_t server_state,
  on{substate_of(server_state)},
  off{initial_substate_of(server_state)};
}
\end{minted}

\subsection{event(\textit{Funktion})}
Events bilden das Herzstück von SObjectizer-States. Wenn ein Element in der Inbox ankommt was
mit dem Typ übereinstimmt, welches die übergebene Funktion als Parameter besitzt wird dieses Event
aktiviert.

\begin{minted}{c++}

struct msg_container {
  int sender_id;
  string msg;
}

class test final : public so_5::agent_t{
public:

  void so_define_agent() override {
  
    server_state.event(&event_handler);
  
  }

private:
  state_t server_state,
  
  void event_handler(mhood_t<msg_container> msg_c) {
    cout << "Nachricht von " << msg_c->sender_id << ": " << msg_c->msg << endl;
  }
}
\end{minted}

\section{Typen/Klassen}

\subsection{so\_5::environment\_t}
Stellt die Umgebung bereit in welcher die Agents agieren können.

\subsection{so\_5::coop\_t}
so\_5::coop\_t stellt den Kontext bereit, mithilfe dessen Agents erstellt werden können.

\subsection{so\_5::mbox\_t}
Dieser Typ ist die eigentliche Inbox/Adresse mithilfe der sich die Agents untereinander verständigen.

\subsection{so\_5::agent\_t}
Von dieser Klasse muss geerbt werden, wenn ein Agent erstellt werden soll.

\subsection{so\_5::signal\_t}
Ist ein vereinfachtes struct welches ohne Parameter abeschickt werden kann. Wie der Name bereits sagt können
zwischen Agents Signale herumgeschickt werden, welche für Events als Indikator dienen können.

\subsection{state\_t}
Das Actor-Modell für welches SObjectizer ein Framework ist kennt sogenannte States. Ein Agent verhält sich
unterschiedlich je nach dem in welchem State er gerade ist.
States können auf folgende weise erstellt werden:
\begin{minted}{c++}
state_t server_state{this};
\end{minted}
Dabei ist State eine Member-Variable der erstellten Agent-Klasse.
Um das Event-Handling zu vereinfachen können States auch von States ''erben''.
Das kann mit ''substate\_of()'' und ''initial\_substate\_of()'' erreicht werden.
\begin{minted}{c++}
state_t server_state{this},
on{substate_of(server_state)},
off{initial_substate_of(server_state)};
\end{minted}
Der Unterschied zwischen diesen beiden Funktionen ist, dass bei ''initial\_substate\_of()'' der erstellte State
derjenige ist, in welchem sich der Agent initial befindet.
