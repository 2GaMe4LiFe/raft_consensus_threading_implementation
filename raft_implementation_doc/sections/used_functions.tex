\chapter{Benutzte Sobjektizer-Funktionen}

\vspace{10mm}

SObjectizer besitzt an sich eigentlich eine Dokumentation. Leider ist diese
nicht sonderlich detailreich und in manchen Fällen auch nicht aufschlussreich.
Der einzige Weg herauszufinden wie etwas funktioniert war es in einem Beispiel auszuprobieren.

\section{Funktionen}

\subsection{so\_5::launch()}
Diese Funktion ist essentiell für die Verwaltung von Agents. Prinzipiell wird nur ein Argument übernommen.
Dieses ist eine Funktion welche eine so\_5::environment\_t Instanz verwaltet. Auf Basis dieser Instanz werden
die Agents angelegt und verwaltet. Erst wenn diese Umgebung beendet wird wird die launch-Funktion beendet.

\subsection{introduce\_coop()}
introduce\_coop() ist eine Member-Function von so\_5::environment\_t. Diese Funktion wird in der Dokumentation von
SObjectizer nicht erklärt. Sie wird einfach als Code-Snippet als notwendiges Übel bereitgestellt. Der einzige
Parameter ist wieder wie in so\_5::launch() eine Funktion welche sich wiederum um das eigentliche Anlegen der
Agents kümmert.

\section{Typen/Klassen}

\subsection{so\_5::mbox\_t}
Dieser Typ ist die eigentliche Inbox/Adresse mithilfe der sich die Agents untereinander verständigen.
